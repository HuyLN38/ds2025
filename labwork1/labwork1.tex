\documentclass{article}

\usepackage{amsmath}
\usepackage{amssymb}
\usepackage{graphicx}
\usepackage{listings}
\usepackage{xcolor}

\title{TCP File Transfer System: Architecture and Implementation}
\author{Technical Documentation}
\date{\today}

\begin{document}

\maketitle

\section{Introduction}
The system implements a client-server architecture for file transfer over TCP/IP. It consists of three main components:

\[
\text{Components} = \{\text{Client}, \text{Server}, \text{Script Interface}\}
\]

\section{System Architecture}
The system follows a traditional client-server architecture with the following mathematical representation:

\begin{align*}
S &= \text{Server Process} \\
C &= \text{Client Process} \\
F &= \text{File Data} \\
P &= \text{Protocol Operations} = \{\text{SEND}, \text{REQUEST}, \text{LIST}\}
\end{align*}

\subsection{Communication Protocol}
The protocol follows a state machine model where each operation $p \in P$ transitions through states:

\begin{align*}
\text{State}_0 &\xrightarrow{\text{Connection}} \text{State}_1 \\
\text{State}_1 &\xrightarrow{\text{Operation Selection}} \text{State}_2 \\
\text{State}_2 &\xrightarrow{\text{Data Transfer}} \text{State}_3 \\
\text{State}_3 &\xrightarrow{\text{Completion}} \text{State}_0
\end{align*}

\section{Key Components}

\subsection{FileTransferServer}
The server implements a listening socket with the following configuration:

\[
\text{Server Socket} = \begin{cases}
\text{Host: } & 0.0.0.0 \\
\text{Port: } & 12345 \\
\text{Backlog: } & 1 \\
\text{Protocol: } & \text{TCP}
\end{cases}
\]

\subsection{FileTransferClient}
The client implements three main operations:

\[
\text{Client Operations} = \begin{cases}
\text{send\_file}(f) & : \text{Upload file } f \text{ to server} \\
\text{request\_file}(f) & : \text{Download file } f \text{ from server} \\
\text{list\_files}() & : \text{Get server file listing}
\end{cases}
\]

\subsection{Data Transfer Protocol}
For file transfers, the protocol follows this sequence:

\begin{align*}
\text{Step 1} &: \text{ Initialize connection} \\
\text{Step 2} &: \text{ Send operation type } p \in P \\
\text{Step 3} &: \text{ Exchange metadata (filename, size)} \\
\text{Step 4} &: \text{ Transfer data in chunks of 4096 bytes} \\
\text{Step 5} &: \text{ Verify completion}
\end{align*}

The progress of file transfer is calculated as:

\[
\text{Progress}(\%) = \frac{\text{Bytes Transferred}}{\text{Total Bytes}} \times 100
\]

\section{Implementation Details}

\subsection{Buffer Sizes}
The system uses optimized buffer sizes:

\begin{align*}
\text{Control Messages} &: 1024 \text{ bytes} \\
\text{Data Transfer} &: 4096 \text{ bytes}
\end{align*}

\subsection{Error Handling}
The system implements comprehensive error handling with the following categories:

\[
\text{Errors} = \begin{cases}
\text{Connection Failures} \\
\text{File Not Found} \\
\text{Transfer Interruption} \\
\text{Invalid Operations}
\end{cases}
\]

\section{Usage Examples}

\subsection{Server Mode}
To start the server:

\[
\text{python script.py --mode server --port 12345}
\]

\subsection{Client Mode}
For client operations:

\begin{align*}
\text{Send} &: \text{ python script.py --mode client --sor send --fn file.txt} \\
\text{Receive} &: \text{ python script.py --mode client --sor receive --fn file.txt} \\
\text{List} &: \text{ python script.py --mode client --sor list}
\end{align*}

\section{Performance Characteristics}
The system's performance can be characterized by:

\[
\text{Transfer Time} = \frac{\text{File Size}}{\text{Network Bandwidth}} + \text{Protocol Overhead}
\]

Where protocol overhead includes:

\[
\text{Overhead} = \text{Connection Setup} + \text{Metadata Exchange} + \text{Acknowledgments}
\]

\section{Code Structure}

\subsection{Class Hierarchy}
The system is organized into two main classes with their interfaces:

\begin{align*}
\text{FileTransferServer} &\rightarrow \{\text{run}, \text{handle\_file\_transfer}, \text{send\_file}, \text{receive\_file}\} \\
\text{FileTransferClient} &\rightarrow \{\text{send\_file}, \text{request\_file}, \text{list\_files}\}
\end{align*}

\subsection{Protocol Flow}
The protocol flow can be represented as a sequence:

\[
\begin{pmatrix}
\text{Client} & \xrightarrow{\text{Connect}} & \text{Server} \\
\text{Client} & \xrightarrow{\text{Operation}} & \text{Server} \\
\text{Server} & \xrightarrow{\text{ACK}} & \text{Client} \\
\text{Client/Server} & \xrightarrow{\text{Data}} & \text{Server/Client}
\end{pmatrix}
\]

\end{document}