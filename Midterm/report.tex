\documentclass[11pt]{article}
\usepackage{graphicx}
\usepackage{listings}
\usepackage{hyperref}
\usepackage{amsmath}
\usepackage{enumitem}
\usepackage{url}
\usepackage{color}
\usepackage{listings}

\lstset{
    basicstyle=\ttfamily\small,
    breaklines=true,
    frame=single,
    language=Python
}

\title{Implementation of a Fault-Tolerant Key-Value\\Server Using RPC: A Redis Clone\\Comprehensive Report}
\author{Technical Report}
\date{December 31, 2024}

\begin{document}

\maketitle

\section{Introduction}
This report details the implementation of a fault-tolerant key-value server that replicates core Redis functionality. The system emphasizes reliability and data consistency while providing a comprehensive set of data structures and operations through a Remote Procedure Call (RPC) interface.

\section{System Architecture}

\subsection{Core Components}
The system consists of four main components:

\begin{enumerate}
    \item \textbf{RPC Server}
    \begin{itemize}
        \item Handles client connections and method invocations
        \item Uses thread pools for concurrent client handling
        \item Manages method registration and execution
    \end{itemize}

    \item \textbf{RPC Client}
    \begin{itemize}
        \item Provides interface for client-server communication
        \item Handles connection management
        \item Implements JSON-based protocol
    \end{itemize}

    \item \textbf{Redis Clone Implementation}
    \begin{itemize}
        \item Core key-value store functionality
        \item Thread-safe operations
        \item Data persistence mechanisms
    \end{itemize}

    \item \textbf{Client Interface}
    \begin{itemize}
        \item Command-line interface for user interaction
        \item Support for all Redis-like commands
        \item Error handling and validation
    \end{itemize}
\end{enumerate}

\subsection{Communication Protocol}
\begin{itemize}
    \item JSON-based RPC protocol
    \item Request Format: \texttt{(method\_name, args, kwargs)}
    \item Response Format: JSON-encoded return value
    \item Transport: TCP/IP sockets with fixed buffer size (1024 bytes)
\end{itemize}

\section{Data Structures and Operations}

\subsection{Key-Value Operations}
Basic operations supported:
\begin{itemize}
    \item \texttt{SET key value [EX seconds]}: Store key-value pair with optional  \\ expiry
    \item \texttt{GET key}: Retrieve value
    \item \texttt{DELETE key}: Remove key-value pair
    \item \texttt{KEYS}: List all keys
    \item \texttt{FLUSHALL}: Clear all data
    \item \texttt{APPEND key value}: Append to existing string value
    \item \texttt{EXISTS key}: Check key existence
\end{itemize}

\subsection{Time-To-Live (TTL) Operations}
\begin{itemize}
    \item \texttt{EXPIRE key seconds}: Set key expiration time
    \item \texttt{TTL key}: Get remaining time
    \begin{itemize}
        \item Returns -2 if key doesn't exist
        \item Returns -1 if key has no expiration
    \end{itemize}
    \item \texttt{PERSIST key}: Remove expiration from key
\end{itemize}

\subsection{Hash Operations}
\begin{itemize}
    \item \texttt{HSET hash\_key field value}: Set field in hash
    \item \texttt{HGET hash\_key field}: Get field value
    \item \texttt{HDEL hash\_key field}: Delete field
    \item \texttt{HGETALL hash\_key}: Get all fields and values
    \item \texttt{HDELALL hash\_key}: Delete entire hash
\end{itemize}

\subsection{Sorted Set Operations}
\begin{itemize}
    \item \texttt{ZSET key score value}: Add element with score
    \item \texttt{ZRANGE key start stop}: Get range (low to high)
    \item \texttt{ZREVRANGE key start stop}: Get range (high to low)
    \item \texttt{ZRANK key value}: Get element rank
    \item \texttt{ZDELVALUE key value}: Remove element
    \item \texttt{ZDELKEY key}: Delete entire sorted set
    \item \texttt{ZGETALL key}: Get all elements with scores
\end{itemize}

\subsection{List Operations}
\begin{itemize}
    \item \texttt{LPUSH/RPUSH key value [value ...]}: Add elements
    \item \texttt{LPOP/RPOP key}: Remove and return element
    \item \texttt{LRANGE key start stop}: Get range of elements
    \item \texttt{LLEN key}: Get list length
    \item \texttt{DELPUSH key}: Clear and reinitialize list
\end{itemize}

\section{Fault Tolerance Mechanisms}

\subsection{Thread Safety}
\begin{itemize}
    \item Reentrant locks (RLock) for data store operations
    \item Separate locks for list operations
    \item Atomic operations for data modifications
    \item Thread-safe method registration and invocation
\end{itemize}

\subsection{Data Persistence}
\begin{itemize}
    \item Periodic snapshots to disk (default 30-second interval)
    \item Background thread for snapshot management
    \item JSON-based snapshot format:
\end{itemize}

\begin{lstlisting}
snapshot_data = {
    'data': self.data_store,
    'expiry': self.expiry_times,
    'sorted_sets': self.sorted_sets 
}
\end{lstlisting}

\subsection{Error Handling}
\begin{itemize}
    \item Comprehensive exception handling for all operations
    \item Logging system with operation tracking
    \item Graceful handling of network failures
    \item Type checking and validation
    \item Automatic cleanup of expired keys
\end{itemize}

\section{Implementation Details}

\subsection{Thread Management}
\begin{lstlisting}
# Server thread pool
Thread(target=self.__handle__, args=[client, address]).start()

# Background threads
self.cleanup_thread = threading.Thread(
    target=self._cleanup_expired_keys, 
    daemon=True
)
self.snapshot_thread = threading.Thread(
    target=self._periodic_snapshot, 
    daemon=True
)
\end{lstlisting}

\subsection{Data Store Structure}
\begin{lstlisting}
class FaultTolerantRedisClone:
    def __init__(self):
        self.data_store: Dict[str, Any] = {}
        self.sorted_sets: Dict[str, List[tuple]] = {} 
        self.expiry_times: Dict[str, float] = {} 
        self.lock = threading.RLock()
        self.lock_list = threading.Lock()
\end{lstlisting}

\section{Performance Considerations}

\subsection{Memory Management}
\begin{itemize}
    \item In-memory storage with disk persistence
    \item Efficient string operations
    \item Memory-conscious data structures
    \item Background cleanup of expired keys
\end{itemize}

\subsection{Network Efficiency}
\begin{itemize}
    \item JSON serialization for data transport
    \item Fixed buffer size management
    \item Connection pooling for multiple clients
\end{itemize}

\section{Reliability Features}

\subsection{Data Integrity}
\begin{itemize}
    \item Atomic operations for data consistency
    \item Snapshot verification on load
    \item Transaction logging
    \item Type validation
\end{itemize}

\subsection{Recovery Mechanisms}
\begin{itemize}
    \item Automatic snapshot recovery
    \item Connection failure handling
    \item Error state recovery
    \item Expired key cleanup
\end{itemize}

\section{Client Interface Example}
\begin{lstlisting}
# Basic operations
Redis Clone Client
Type your commands (e.g., `SET key value`) or type `EXIT` to quit.
> SET mykey distributed
OK
> GET mykey  
distributed
> DEL mykey
distributed
>GET mykey
None


# Hash operations
> HSET student name kious
OK
> HSET student age 19
OK
> HGETALL student
{'name': 'kious', 'age': '19'}

# Sorted set operations
> ZSET scores 100 alice
1
> ZSET scores 20 bob
1
> ZSET scores 50 kious
1
> ZRANGE scores 0 3
['bob', 'kious', 'alice']
>EXIST
Goodbye!

\end{lstlisting}

\section{Future Improvements}

Potential enhancements include:
\begin{itemize}
    \item Data replication for high availability
    \item Support for additional complex data types
    \item Transaction support
    \item Incremental backup system
    \item Connection pooling optimization
    \item Cluster support
\end{itemize}

\section{Conclusion}
Fault-tolerant key-value server implementations provide a powerful and  \\efficient solution to key-value storage needs. Integration of comprehensive data structure support tree safety Database stability and error handling This makes it suitable for production applications that require Redis-like  \\applications. Although not as versatile as Redis itself, it provides a solid foundation for space-hungry distributed systems. Reliable storage and  \\access to data

\end{document}